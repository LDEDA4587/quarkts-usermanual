\subsection{Trace and debugging}
QuarkTS include some basic macros to print out debugging messages. Messages can be simple text or the value of variables in specific base-formats. 
To use the trace macros, a single-char output function must be defined using the \lstinline{qTrace_Set_OutputFcn()} macro.\index{\lstinline{qTrace_Set_OutputFcn}}
\medskip

\begin{lstlisting}[style=CStyle]
qTrace_Set_OutputFcn( qPutChar_t fcn )
\end{lstlisting}

Where \lstinline{fcn} is a pointer to the single-char output function following the prototype:
\medskip

\begin{lstlisting}[style=CStyle]
void SingleChar_OutputFcn( void *sp, const char c ){
    /*
    TODO : print out the c variable using the
    selected peripheral.
    */
}
\end{lstlisting}

The body of this user-defined function should have a hardware-dependent code to print out the \lstinline{c} variable through a specific peripheral.

\subsubsection{Viewing variables}
For viewing or tracing a variable (up to 32-bit data) through debug, one of the following macros are available:
\index{\lstinline{qTrace_Var}} \index{\lstinline{qTrace_Variable}} \index{\lstinline{qDebug_Var}} \index{\lstinline{qDebug_Variable}}
\medskip

\begin{lstlisting}[style=CStyle]
qTrace_Var( Var, DISP_TYPE_MODE )
qTrace_Variable( Var, DISP_TYPE_MODE )
\end{lstlisting}
\begin{lstlisting}[style=CStyle]
qDebug_Var( Var, DISP_TYPE_MODE )
qDebug_Variable( Var, DISP_TYPE_MODE )
\end{lstlisting}

\subsubsection*{Parameters:}
\begin{itemize}
    \item \lstinline{Var} : The target variable. 
    \item \lstinline{DISP_TYPE_MODE } :  Visualization mode. It must be one of the following parameters(case sensitive): \lstinline{Bool}, \lstinline{Float}, \lstinline{Binary}, \lstinline{Octal}, \lstinline{Decimal}, \lstinline{Hexadecimal}, \lstinline{UnsignedBinary}, \lstinline{UnsignedOctal}, \lstinline{UnsignedDecimal}, \lstinline{UnsignedHexadecimal}. 
\end{itemize}

The only difference between \lstinline{qTrace_} and  \lstinline{Debug}, is that \lstinline{qTrace_} macros, print out additional information provided by the \lstinline{__FILE__}, \lstinline{__LINE__} and \lstinline{__func__} built-in preprocessing macros, mostly available in common C compilers. 

\subsubsection{Viewing a memory block}
For tracing memory from a specified target address, one of the following macros are available: \index{\lstinline{qTrace_Mem}} \index{\lstinline{qTrace_Memory}}

\begin{lstlisting}[style=CStyle]
qTrace_Mem( Pointer, BlockSize )
qTrace_Memory( Pointer, BlockSize )
\end{lstlisting}

\subsubsection*{Parameters:}
\begin{itemize}
    \item \lstinline{Pointer} : The target memory address.
    \item \lstinline{Size} : Number of bytes to be visualized.
\end{itemize}

Hexadecimal notation it's used to format the output of these macros.

\subsubsection{Usage}

In the example below, an UART output function is coded to act as the printer. Here, the target MCU is an ARM-Cortex M0 with the UART1 as the selected peripheral for this purpose.
\medskip

\begin{lstlisting}[style=CStyle]
void putUART1( void *sp, const char c ){
    /* hardware specific code */
    UART1_D = c;
    while ( !(UART1_S1 & UART_S1_TC_MASK) ) {} /*wait until TX is done*/ 
}
\end{lstlisting}  

As seen above, the function follows the required prototype. Later, in the main thread, a call to the \lstinline{qSetDebugFcn()} is used to set up the output-function.

\begin{lstlisting}[style=CStyle]
int main( void ){
   qTrace_Set_OutputFcn( putUART1 );
   ... 
   ...
}
\end{lstlisting}  

After that, trace macros will be available for use.

\begin{lstlisting}[style=CStyle]
void IO_TASK_Callback( qEvent_t e ){
   static qUINT32_t Counter = 0;
   float Sample;
   ...
   ... 
   qTrace_Message( "IO TASK running..." );
   Counter++;
   qTrace_Variable( Counter, UnsignedDecimal );
   Sample = SensorGetSample();
   qTrace_Variable( Sample, Float );
   ...
   ...
}
\end{lstlisting}