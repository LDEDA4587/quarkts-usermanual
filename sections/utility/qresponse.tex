\subsection{Response handler}

An API to simplify the handling of requested responses from terminal interfaces.
\medskip

\begin{lstlisting}[style=CStyle]
qBool_t qResponse_Setup( qResponse_t * const obj, char *xLocBuff, 
                      size_t nMax )
\end{lstlisting}

Initialize the instance of the response handler object. \index{\lstinline{qResponse_Setup}}

\subsubsection*{Parameters:}
\begin{itemize}
    \item \lstinline{obj} : A pointer to the response handler object
    \item \lstinline{xLocBuff} : A pointer to the memory block where the desired response will remain.
    \item \lstinline{nMax} : The size of memory block pointed by \lstinline{xLocBuff}
\end{itemize}

\subsubsection*{Return value:}
\lstinline{qTrue} on success, otherwise returns \lstinline{qFalse}.

\noindent\hrulefill

\begin{lstlisting}[style=CStyle]
void qResponse_Reset( qResponse_t * const obj )
\end{lstlisting}

Reset the response handler. \index{\lstinline{qResponse_Reset}}

\subsubsection*{Parameters:}
\begin{itemize}
    \item \lstinline{obj} : A pointer to the response handler object
\end{itemize}

\noindent\hrulefill

\begin{lstlisting}[style=CStyle]
qBool_t qResponse_Received( qResponse_t * const obj, 
                            const char *Pattern, size_t n )
\end{lstlisting}

Non-blocking response check. \index{\lstinline{qResponse_Received}}

\subsubsection*{Parameters:}
\begin{itemize}
    \item \lstinline{obj} : A pointer to the response handler object.
    \item \lstinline{Pattern} : The data to be checked in the receiver ISR
    \item \lstinline{n} : The length of the data pointer by \lstinline{Pattern} . If \lstinline{Pattern} its string, set \lstinline{n} to zero(0) to auto-compute the length.
\end{itemize}

\subsubsection*{Return value:}
\lstinline{qTrue} if there is a response acknowledge, otherwise returns \lstinline{qFalse}.

\noindent\hrulefill

\begin{lstlisting}[style=CStyle]
qBool_t qResponse_ReceivedWithTimeout( qResponse_t * const obj, 
                                       const char *Pattern, 
                                       size_t n, qTime_t t )
\end{lstlisting}

Non-blocking response check with timeout. \index{\lstinline{qResponse_ReceivedWithTimeout}}

\subsubsection*{Parameters:}
\begin{itemize}
    \item \lstinline{obj} : A pointer to the response handler object.
    \item \lstinline{Pattern} : The data to be checked in the receiver ISR
    \item \lstinline{n} : The length of the data pointer by \lstinline{Pattern} . If \lstinline{Pattern} its string, set \lstinline{n} to zero(0) to auto-compute the length.
    \item \lstinline{obj} : The timeout value in seconds.
\end{itemize}

\subsubsection*{Return value:}
\lstinline{qTrue} if there is a response acknowledge and \lstinline{qTimeoutReached} if timeout expires,  otherwise returns \lstinline{qFalse}.

\noindent\hrulefill

\begin{lstlisting}[style=CStyle]
qBool_t qResponse_ISRHandler( qResponse_t * const obj, 
                              const char rxchar )
\end{lstlisting}

ISR receiver for the response handler. \index{\lstinline{qResponse_ISRHandler}}. This call should be placed inside the ISR in charge of receiving the incoming data (one byte at the time).

\subsubsection*{Parameters:}
\begin{itemize}
    \item \lstinline{obj} : A pointer to the response handler object.
    \item \lstinline{rxchar} : The byte-data from the receiver.
\end{itemize}

\subsubsection*{Return value:}
\lstinline{qTrue} when the response handler object match the request from \lstinline{qResponse_Received()}.