\subsection{Byte sized buffers}

An interrupt-safe byte-sized ring buffer.
\medskip

\begin{lstlisting}[style=CStyle]
qBool_t qBSBuffer_Setup( qBSBuffer_t * const obj, volatile qUINT8_t *buffer, 
                      const size_t length )
\end{lstlisting}

Initialize the byte-sized buffer. \index{\lstinline{qBSBuffer_Setup}}

\subsubsection*{Parameters:}
\begin{itemize}
    \item \lstinline{obj} : A pointer to the byte-sized buffer object
    \item \lstinline{buffer} : Block of memory or array of data.
    \item \lstinline{length} : The size of \lstinline{buffer}(Must be a power of two)
\end{itemize}

\noindent\hrulefill

\begin{lstlisting}[style=CStyle]
qBool_t qBSBuffer_Put( qBSBuffer_t * const obj, const qUINT8_t data )
\end{lstlisting}

Adds an element of data to the byte-sized buffer. \index{\lstinline{qBSBuffer_Put}}

\subsubsection*{Parameters:}
\begin{itemize}
    \item \lstinline{obj} : A pointer to the byte-sized buffer object
    \item \lstinline{data} : The data to be added.
\end{itemize}

\subsubsection*{Return value:}
\lstinline{qTrue} on success, otherwise returns \lstinline{qFalse}.

\noindent\hrulefill

\begin{lstlisting}[style=CStyle]
qBool_t qBSBuffer_Read( qBSBuffer_t * const obj, void *dest, 
                        const size_t n )
\end{lstlisting}

Gets \lstinline{n} data from the byte-sized buffer and removes them. \index{\lstinline{qBSBuffer_Read}}

\subsubsection*{Parameters:}
\begin{itemize}
    \item \lstinline{obj} : A pointer to the byte-sized buffer object
    \item \lstinline{dest} : The location where the data will be written.
\end{itemize}

\subsubsection*{Return value:}
\lstinline{qTrue} on success, otherwise returns \lstinline{qFalse}.

\noindent\hrulefill

\begin{lstlisting}[style=CStyle]
qBool_t qBSBuffer_Get( qBSBuffer_t * const obj, qUINT8_t *dest )
\end{lstlisting}

Gets one data-byte from the front of the byte-sized buffer, and remove it. \index{\lstinline{qBSBuffer_Get}}

\subsubsection*{Parameters:}
\begin{itemize}
    \item \lstinline{obj} : A pointer to the byte-sized buffer object
    \item \lstinline{dest} : The location where the data will be written.
\end{itemize}

\subsubsection*{Return value:}
\lstinline{qTrue} on success, otherwise returns \lstinline{qFalse}.


\noindent\hrulefill

\begin{lstlisting}[style=CStyle]
qUINT8_t qBSBuffer_Peek( const qBSBuffer_t * const obj )
\end{lstlisting}

Looks for one byte from the head of the byte-sized buffer without removing it. \index{\lstinline{qBSBuffer_Peek}}

\subsubsection*{Parameters:}
\begin{itemize}
    \item \lstinline{obj} : A pointer to the byte-sized buffer object
\end{itemize}

\subsubsection*{Return value:}
Byte of data, or zero if nothing in the buffer.

\noindent\hrulefill

\begin{lstlisting}[style=CStyle]
qBool_t qBSBuffer_Empty( const qBSBuffer_t * const obj )
\end{lstlisting}

Query the empty status of the byte-sized buffer. \index{\lstinline{qBSBuffer_Empty}}

\subsubsection*{Parameters:}
\begin{itemize}
    \item \lstinline{obj} : A pointer to the byte-sized buffer object
\end{itemize}

\subsubsection*{Return value:}
\lstinline{qTrue} if the byte-sized buffer is empty, \lstinline{qFalse} if it is not.


\noindent\hrulefill

\begin{lstlisting}[style=CStyle]
qBool_t qBSBuffer_IsFull( const qBSBuffer_t * const obj )
\end{lstlisting}

Query the full status of the byte-sized buffer. \index{\lstinline{qBSBuffer_IsFull}}

\subsubsection*{Parameters:}
\begin{itemize}
    \item \lstinline{obj} : A pointer to the byte-sized buffer object
\end{itemize}

\subsubsection*{Return value:}
\lstinline{qTrue} if the byte-sized buffer is full, \lstinline{qFalse} if it is not.


\noindent\hrulefill

\begin{lstlisting}[style=CStyle]
size_t qBSBuffer_Count( const qBSBuffer_t * const obj )
\end{lstlisting}

Query the number of elements in the byte-sized buffer. \index{\lstinline{qBSBuffer_Count}}

\subsubsection*{Parameters:}
\begin{itemize}
    \item \lstinline{obj} : A pointer to the byte-sized buffer object.
\end{itemize}

\subsubsection*{Return value:}
Number of elements in the byte-sized buffer.
